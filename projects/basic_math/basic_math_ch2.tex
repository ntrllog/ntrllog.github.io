\documentclass[12pt]{article}
\usepackage{amsthm, amsmath}

\renewcommand{\baselinestretch}{2}

\title{Chapter 2}
\author{}
\date{}

\begin{document}

\maketitle
\newtheorem{theorem}{Theorem}
\newtheorem{corollary}{Corollary}[theorem]

\stepcounter{section}
\section{Linear Equations}
\subsection{Equations In Two Unknowns}
\subsection*{Exercises}
Solve the following systems of equations for $x$ and $y$. \\
\\
1. $\begin{aligned}[t]
2x-y=3 \\
x+y=2
\end{aligned}$ \\
\\
Adding the two equations will cancel out the $y$, giving us
\begin{align*}
2x+x&=3+2 \\
3x&=5 \\
x&=\displaystyle \frac{5}{3}.
\end{align*}
Using this value for $x$ in the second equation gives us
\begin{align*}
\displaystyle \frac{5}{3}+y&=2 \\
y&=2-\displaystyle \frac{5}{3} \\
y&=\displaystyle \frac{1}{3}.
\end{align*}
2. $\begin{aligned}[t]
-4x+7y=-1 \\
x-2y=-4
\end{aligned}$ \\
\\
Multiplying the second equation by 4 gives us
\begin{align*}
-4x+7y&=-1 \\
4x-8y&=-16.
\end{align*}
Adding the two equations will cancel out the $x$, giving us
\begin{align*}
7y-8y&=-1-16 \\
-y&=-17 \\
y&=17.
\end{align*}
Using this value for $y$ in the second equation gives us
\begin{align*}
x-2(17)&=-4 \\
x-34&=-4 \\
x&=30.
\end{align*}
3. $\begin{aligned}[t]
3x+4y&=-2 \\
-2x-3y&=1
\end{aligned}$ \\
\\
Multiplying the first equation by 2 and the second equation by 3 gives us
\begin{align*}
6x+8y&=-4 \\
-6x-9y&=3.
\end{align*}
Adding the two equations will cancel out the $x$, giving us
\begin{align*}
8y-9y&=-4+3 \\
-y&=-1 \\
y&=1.
\end{align*}
Using this value for $y$ in the second equation gives us
\begin{align*}
-2x-3(1)&=1 \\
-2x-3&=1 \\
-2x&=4 \\
x&=-2.
\end{align*}
4. $\begin{aligned}[t]
-3x+2y&=-1 \\
x-y&=2
\end{aligned}$ \\
\\
Multiplying the second equation by 3 gives us
\begin{align*}
-3x+2y&=-1 \\
3x-3y&=6.
\end{align*}
Adding the two equations will cancel out the $x$, giving us
\begin{align*}
2y-3y&=-1+6 \\
-y&=5 \\
y&=-5.
\end{align*}
Using this value for $y$ in the second equation gives us
\begin{align*}
x-(-5)&=2 \\
x+5&=2 \\
x&=-3.
\end{align*}
5. $\begin{aligned}[t]
-3x+y&=0 \\
x-y&=1
\end{aligned}$ \\
\\
Adding the two equations will cancel out the $y$, giving us
\begin{align*}
-3x+x&=0+1 \\
-2x&=1 \\
x&=-\displaystyle \frac{1}{2}.
\end{align*}
Using this value for $x$ in the second equation gives us
\begin{align*}
-\displaystyle \frac{1}{2}-y&=1 \\
-y&=1+\displaystyle \frac{1}{2} \\
-y&=\displaystyle \frac{3}{2} \\
y&=-\displaystyle \frac{3}{2}.
\end{align*}
6. $\begin{aligned}[t]
3x+7y&=0 \\
x-y&=0
\end{aligned}$ \\
\\
Multiplying the second equation by 7 gives us
\begin{align*}
3x+7y&=0 \\
7x-7y&=0.
\end{align*}
Adding the two equations will cancel out the $y$, giving us
\begin{align*}
3x+7x&=0+0 \\
10x&=0 \\
x&=0.
\end{align*}
Using this value for $x$ in the second equation gives us
\begin{align*}
0-y&=0 \\
-y&=0 \\
y&=0.
\end{align*}
7. $\begin{aligned}[t]
7x-y&=2 \\
2x+2y&=4
\end{aligned}$ \\
\\
Multiplying the first equation by 2 gives us
\begin{align*}
14x-2y&=4 \\
2x+2y&=4.
\end{align*}
Adding the two equations will cancel out the $y$, giving us
\begin{align*}
14x+2x&=4+4 \\
16x&=8 \\
x&=\displaystyle \frac{8}{16} \\
x&=\displaystyle \frac{1}{2}.
\end{align*}
Using this value for $x$ in the second equation gives us
\begin{align*}
2\bigg(\frac{1}{2}\bigg)+2y&=4 \\
1+2y&=4 \\
2y&=3 \\
y&=\displaystyle \frac{3}{2}.
\end{align*}
8. $\begin{aligned}[t]
-4x-7y&=5 \\
2x+y&=6
\end{aligned}$ \\
\\
Multiplying the second equation by 7 gives us
\begin{align*}
-4x-7y&=5 \\
14x+7y&=42.
\end{align*}
Adding the two equations will cancel out the $y$, giving us
\begin{align*}
-4x+14x&=5+42 \\
10x&=47 \\
x&=\displaystyle \frac{47}{10}.
\end{align*}
Using this value for $x$ in the second equation gives us
\begin{align*}
2\bigg(\displaystyle \frac{47}{10}\bigg)+y&=6 \\
\displaystyle \frac{94}{10}+y&=6 \\
y&=6-\displaystyle \frac{94}{10} \\
y&=\displaystyle \frac{60}{10}-\displaystyle \frac{94}{10} \\
y&=-\displaystyle \frac{34}{10} \\
y&=-\displaystyle \frac{17}{5}.
\end{align*}
9. Let $a$, $b$, $c$, $d$ be numbers such that $ad-bc\neq0$. Solve the following systems of equations for $x$ and $y$ in terms of $a$, $b$, $c$, $d$. \\
\\
a) $\begin{aligned}[t]
ax+by&=1 \\
cx+dy&=2
\end{aligned}$ \\
\\
Multiplying the first equation by $d$ and the second equation by $-b$ gives us
\begin{align*}
adx+bdy&=d \\
-bcx-bdy&=-2b.
\end{align*}
Adding the two equations will cancel out the $y$, giving us
\begin{align*}
adx-bcx&=d-2b \\
x(ad-bc)&=d-2b \\
x&=\displaystyle \frac{d-2b}{ad-bc}.
\end{align*}
Instead of using this value for $x$ in one of the equations, we can do the same thing as above to find $y$. \\
Multiplying the first equation by $-c$ and the second equation by $a$ gives us
\begin{align*}
-acx-bcy&=-c \\
acx+ady&=2a.
\end{align*}
Adding the two equations will cancel out the $x$, giving us
\begin{align*}
-bcy+ady&=-c+2a \\
y(-bc+ad)&=-c+2a \\
y&=\displaystyle \frac{2a-c}{ad-bc}.
\end{align*}
b) $\begin{aligned}[t]
ax+by&=3 \\
cx+dy&=-4
\end{aligned}$ \\
\\
Multiplying the first equation by $d$ and the second equation by $-b$ gives us
\begin{align*}
adx+bdy&=3d \\
-bcx-bdy&=4b.
\end{align*}
Adding the two equations will cancel out the $y$, giving us
\begin{align*}
adx-bcx&=3d+4b \\
x(ad-bc)&=3d+4b \\
x&=\displaystyle \frac{3d+4b}{ad-bc}.
\end{align*}
Instead of using this value for $x$ in one of the equations, we can do the same thing as above to find $y$. \\
Multiplying the first equation by $-c$ and the second equation by $a$ gives us
\begin{align*}
-acx-bcy&=-3c \\
acx+ady&=-4a.
\end{align*}
Adding the two equations will cancel out the $x$, giving us
\begin{align*}
-bcy+ady&=-3c-4a \\
y(-bc+ad)&=-3c-4a \\
y&=\displaystyle \frac{-4a-3c}{ad-bc}.
\end{align*}
c) $\begin{aligned}[t]
ax+by&=-2 \\
cx+dy&=3
\end{aligned}$ \\
\\
Multiplying the first equation by $d$ and the second equation by $-b$ gives us
\begin{align*}
adx+bdy&=-2d \\
-bcx-bdy&=-3b.
\end{align*}
Adding the two equations will cancel out the $y$, giving us
\begin{align*}
adx-bcx&=-2d-3b \\
x(ad-bc)&=-2d-3b \\
x&=\displaystyle \frac{-2d-3b}{ad-bc}.
\end{align*}
Instead of using this value for $x$ in one of the equations, we can do the same thing as above to find $y$. \\
Multiplying the first equation by $-c$ and the second equation by $a$ gives us
\begin{align*}
-acx-bcy&=2c \\
acx+ady&=3a.
\end{align*}
Adding the two equations will cancel out the $x$, giving us
\begin{align*}
-bcy+ady&=2c+3a \\
y(-bc+ad)&=2c+3a \\
y&=\displaystyle \frac{3c+3a}{ad-bc}.
\end{align*}
d) $\begin{aligned}[t]
ax+by&=5 \\
cx+dy&=7
\end{aligned}$ \\
\\
Multiplying the first equation by $d$ and the second equation by $-b$ gives us
\begin{align*}
adx+bdy&=5d \\
-bcx-bdy&=-7b.
\end{align*}
Adding the two equations will cancel out the $y$, giving us
\begin{align*}
adx-bcx&=5d-7b \\
x(ad-bc)&=5d-7b \\
x&=\displaystyle \frac{5d-7b}{ad-bc}.
\end{align*}
Instead of using this value for $x$ in one of the equations, we can do the same thing as above to find $y$. \\
Multiplying the first equation by $-c$ and the second equation by $a$ gives us
\begin{align*}
-acx-bcy&=-5c \\
acx+ady&=7a.
\end{align*}
Adding the two equations will cancel out the $x$, giving us
\begin{align*}
-bcy+ady&=-5c+7a \\
y(-bc+ad)&=-5c+7a \\
y&=\displaystyle \frac{7a-5c}{ad-bc}.
\end{align*}
10. Making the same assumptions as in Exercise 9, show that the solution of the system
\begin{center}
$ax+by=0$, \\
$cx+dy=0$
\end{center}
must be $x=0$ and $y=0$. \\
\\
Multiplying the first equation by $d$ and the second equation by $-b$ gives us
\begin{align*}
adx+bdy&=0 \\
-bcx-bdy&=0.
\end{align*}
Adding the two equations will cancel out the $y$, giving us
\begin{align*}
adx-bcx&=0 \\
x(ad-bc)&=0 \\
x&=\displaystyle \frac{0}{ad-bc} \\
x&=0.
\end{align*}
We can't use this value for $x$ in the second equation (or the first, but it's the same situation anyway). Let's see what happens. 
\begin{align*}
c(0)+dy&=0 \\
0+dy&=0 \\
dy&=0
\end{align*}
We can't divide both sides by $d$ because we don't know if $d$ is equal to 0 or not. \\
Instead, we'll multiply both equations and cancel out the $x$. Multiplying the first equation by $-c$ and the second equation by $a$ gives us
\begin{align*}
-acx-bcy&=0 \\
acx+ady&=0.
\end{align*}
Adding the two equations will cancel out the $x$, giving us
\begin{align*}
-bcy+ady&=0 \\
y(-bc+ad)&=0 \\
y&=\displaystyle \frac{0}{ad-bc} \\
y&=0.
\end{align*}
11. Let $a$, $b$, $c$, $d$, $u$, $v$ be numbers and assume that $ad-bc\neq0$. Solve the following system of equations for $x$ and $y$ in terms of $a$, $b$, $c$, $d$, $u$, $v$:
\begin{center}
$ax+by=u$ \\
$cx+dy=v.$
\end{center}
Verify that the answer you get is actually a solution. \\
\\
Multiplying the first equation by $-c$ and the second equation by $a$ gives us
\begin{align*}
-acx-bcy&=-cu \\
acx+ady&=av.
\end{align*}
Adding the two equations will cancel out the $x$, giving us
\begin{align*}
-bcy+ady&=-cu+av \\
y(-bc+ad)&=-cu+av \\
y&=\displaystyle \frac{-cu+av}{ad-bc}.
\end{align*}
Multiplying the first equation by $d$ and the second equation by $-b$ gives us
\begin{align*}
adx+bdy&=du \\
-bcx-bdy&=-bv.
\end{align*}
Adding the two equations will cancel out the $y$, giving us
\begin{align*}
adx-bcx&=du-bv \\
x(ad-bc)&=du-bv \\
x&=\displaystyle \frac{du-bv}{ad-bc}.
\end{align*}
To verify that $x$, $y$ are actually solutions, we plug them into the two equations. Starting with the first equation, we get
\begin{align*}
a\bigg(\displaystyle \frac{du-bv}{ad-bc}\bigg)+b\bigg(\displaystyle \frac{-cu+av}{ad-bc}\bigg)&=u \\
\displaystyle \frac{adu-abv}{ad-bc}+\displaystyle \frac{-bcu+abv}{ad-bc}&=u \\
\displaystyle \frac{adu-abv-bcu+abv}{ad-bc}&=u \\
\displaystyle \frac{adu-bcu}{ad-bc}&=u \\
\displaystyle \frac{u(ad-bc)}{ad-bc}&=u \\
u&=u.
\end{align*}
For the second equation, we get
\begin{align*}
c\bigg(\displaystyle \frac{du-bv}{ad-bc}\bigg)+d\bigg(\displaystyle \frac{-cu+av}{ad-bc}\bigg)&=v \\
\displaystyle \frac{cdu-bcv}{ad-bc}+\displaystyle \frac{-cdu+adv}{ad-bc}&=v \\
\displaystyle \frac{cdu-bcv-cdu+adv}{ad-bc}&=v \\
\displaystyle \frac{-bcv+adv}{ad-bc}&=v \\
\displaystyle \frac{v(ad-bc)}{ad-bc}&=v \\
v&=v.
\end{align*}
\subsection{Equations In Three Unknowns}
\subsection*{Exercises}
Solve the following equations for $x$, $y$, $z$. \\
\\
1. $\begin{aligned}[t]
2x-3y+z&=0 \\
x+y+z&=1 \\
x-2y-4z&=2
\end{aligned}$ \\
\\
We will reduce this to a system of two equations in two unknowns by eliminating the $z$. \\
We start by looking at the first and second equations. Multiplying the first equation by $-1$ gives us
\begin{align}
-2x+3y-z&=0 \\
x+y+z&=1 \\
x-2y-4z&=2
\end{align}
Adding (1) and (2) will cancel out the $z$, giving us
\begin{align}
-2x+x+3y+y&=0+1 \\
-x+4y&=1
\end{align}
Now we need another equation in two unknowns, so we go back to the beginning and eliminate the $z$, this time looking at the first and third equations. Multiplying the first equation by 4 gives us
\begin{align}
8x-12y+4z&=0 \\
x+y+z&=1 \\
x-2y-4z&=2
\end{align}
Adding (6) and (8) will cancel out the $z$, giving us
\begin{align}
8x+x-12y-2y&=0+2 \\
9x-14y&=2
\end{align}
Now we have a system of two equations (from (5) and (10))
\begin{align}
-x+4y&=1 \\
9x-14y&=2
\end{align}
Multiplying (11) by 9 gives us
\begin{align*}
-9x+36y&=9 \\
9x-14y&=2
\end{align*}
Adding the two equations will cancel out the $x$, giving us
\begin{align*}
36y-14y&=9+2 \\
22y&=11 \\
y&=\displaystyle \frac{11}{22} \\
y&=\displaystyle \frac{1}{2}
\end{align*}
Using this value for $y$ in (11) gives us
\begin{align*}
-x+4\bigg(\displaystyle \frac{1}{2}\bigg)&=1 \\
-x+2&=1 \\
-x&=-1 \\
x&=1
\end{align*}
Using these values for $x$ and $y$ in the second equation (from the beginning) gives us
\begin{align*}
1+\displaystyle \frac{1}{2}+z&=1 \\
\displaystyle \frac{1}{2}+z&=0 \\
z&=-\displaystyle \frac{1}{2}
\end{align*}
\setcounter{equation}{0}
2. $\begin{aligned}[t]
2x-y+z&=1 \\
4x+y+z&=2 \\
x-y-2z&=0
\end{aligned}$ \\
\\
We will reduce this to a system of two equations in two unknowns by eliminating the $y$. \\
We start by looking at the first and second equations. Adding the first and second equation will cancel out the $y$, giving us
\begin{align}
2x+4x+z+z&=1+2 \\
6x+2z&=3
\end{align}
Now we need another equation in two unknowns, so we go back to the beginning and eliminate the $y$, this time looking at the second and third equations. Adding the second and third equation will cancel out the $y$, giving us
\begin{align}
4x+x+z-2z&=2+0 \\
5x-z&=2
\end{align}
Now we have a system of two equations (from (2) and (4))
\begin{align}
6x+2z&=3 \\
5x-z&=2
\end{align}
Multiplying (6) by 2 gives us
\begin{align*}
6x+2z&=3 \\
10x-2z&=4
\end{align*}
Adding the two equations will cancel out the $z$, giving us
\begin{align*}
6x+10x&=3+4 \\
16x&=7 \\
x&=\displaystyle \frac{7}{16}
\end{align*}
Using this value for $x$ in (6) gives us
\begin{align*}
5\bigg(\displaystyle \frac{7}{16}\bigg)-z&=2 \\
\displaystyle \frac{35}{16}-z&=2 \\
-z&=2-\displaystyle \frac{35}{16} \\
-z&=\displaystyle \frac{32}{16}-\displaystyle \frac{35}{16} \\
-z&=-\displaystyle \frac{3}{16} \\
z&=\displaystyle \frac{3}{16}
\end{align*}
Using these values for $x$ and $z$ in the first equation (from the beginning) gives us
\begin{align*}
2\bigg(\displaystyle \frac{7}{16}\bigg)-y+\displaystyle \frac{3}{16}&=1 \\
\displaystyle \frac{14}{16}-y+\displaystyle \frac{3}{16}&=1 \\
-y+\displaystyle \frac{17}{16}&=1 \\
-y&=1-\displaystyle \frac{17}{16} \\
-y&=\displaystyle \frac{16}{16}-\displaystyle \frac{17}{16} \\
-y&=-\displaystyle \frac{1}{16} \\
y&=\displaystyle \frac{1}{16}
\end{align*}
\setcounter{equation}{0}
3. $\begin{aligned}[t]
x+4y-4z&=1 \\
x+2y+z&=2 \\
4x-3y-2z&=1
\end{aligned}$ \\
\\
We will reduce this to a system of two equations in two unknowns by eliminating the $x$. \\
We start by looking at the first and second equations. Multiplying the first equation by $-1$ gives us
\begin{align}
-x-4y+4z&=-1 \\
x+2y+z&=2 \\
4x-3y-2z&=1
\end{align}
Adding (1) and (2) will cancel out the $x$, giving us
\begin{align}
-4y+2y+4z+z&=-1+2 \\
-2y+5z&=1
\end{align}
Now we need another equation in two unknowns, so we go back to the beginning and eliminate the $x$, this time looking at the second and third equations. Multiplying the second equation by $-4$ gives us
\begin{align}
x+4y-4z&=1 \\
-4x-8y-4z&=-8 \\
4x-3y-2z&=1
\end{align}
Adding (7) and (8) will cancel out the $x$, giving us
\begin{align}
-8y-3y-4z-2z&=-8+1 \\
-11y-6z&=-7
\end{align}
Now we have a system of two equations (from (5) and (10))
\begin{align}
-2y+5z&=1 \\
-11y-6z&=-7
\end{align}
Multiplying (11) by 6 and (12) by 5 gives us
\begin{align*}
-12y+30z&=6 \\
-55y-30z&=-35
\end{align*}
Adding the two equations will cancel out the $z$, giving us
\begin{align*}
-12y-55y&=6-35 \\
-67y&=-29 \\
y&=\displaystyle \frac{-29}{-67} \\
y&=\displaystyle \frac{29}{67}
\end{align*}
Using this value for $y$ in (11) gives us
\begin{align*}
-2\bigg(\displaystyle \frac{29}{67}\bigg)+5z&=1 \\
-\displaystyle \frac{58}{67}+5z&=1 \\
5z&=1+\displaystyle \frac{58}{67} \\
5z&=\displaystyle \frac{67}{67}+\displaystyle \frac{58}{67} \\
5z&=\displaystyle \frac{125}{67} \\
\displaystyle \frac{5z}{1}&=\displaystyle \frac{125}{67} \\
5z\cdot67&=125\cdot1 \\
335z&=125 \\
z&=\displaystyle \frac{125}{335} \\
z&=\displaystyle \frac{25}{67}
\end{align*}
Using these values for $y$ and $z$ in the second equation (from the beginning) gives us
\begin{align*}
x+2\bigg(\displaystyle \frac{29}{67}\bigg)+\displaystyle \frac{25}{67}&=2 \\
x+\displaystyle \frac{58}{67}+\displaystyle \frac{25}{67}&=2 \\
x+\displaystyle \frac{83}{67}&=2 \\
x&=2-\displaystyle \frac{83}{67} \\
x&=\displaystyle \frac{134}{67}-\displaystyle \frac{83}{67} \\
x&=\displaystyle \frac{51}{67}
\end{align*}
\setcounter{equation}{0}
4. $\begin{aligned}[t]
x+y+z&=0 \\
x-y+z&=0 \\
2x-y-z&=0
\end{aligned}$ \\
\\
Normally, we would want to reduce this to a system of two equations in two unknowns. However, we can quickly find the value of $x$ by looking at the first and third equations. Adding the first and third equation will cancel out the $y$ (and the $z$), giving us
\begin{align*}
x+2x&=0+0 \\
3x&=0 \\
x&=0
\end{align*}
Now we need an equation in two unknowns (with one of the unknowns being $x$), so we go back to the beginning, this time looking at the second and third equations. \\
Adding the first and second equation will cancel out the $y$, giving us
\begin{align*}
x+x+z+z&=0+0 \\
2x+2z&=0
\end{align*}
Using the value for $x$ we got above gives us
\begin{align*}
2(0)+2z&=0 \\
0+2z&=0 \\
2z&=0 \\
z&=0
\end{align*}
Using these values for $x$ and $z$ in the first equation (from the beginning) gives us
\begin{align*}
0+y+0&=0 \\
y&=0
\end{align*}
5. $\begin{aligned}[t]
5x+3y-z&=0 \\
x+2y+2z&=1 \\
x-2y-2z&=0
\end{aligned}$ \\
\\
Normally, we would want to reduce this to a system of two equations in two unknowns. However, we can quickly find the value of $x$ by looking at the second and third equations. Adding the second and third equation will cancel out the $y$ and $z$, giving us
\begin{align*}
x+x&=1+0 \\
2x&=1 \\
x&=\displaystyle \frac{1}{2}
\end{align*}
Now we need an equation in two unknowns (with one of the unknowns being $x$), so we go back to the beginning, this time looking at the first and second equations. \\
Multiplying the first equation by 2 gives us
\begin{align}
10x+6y-2z&=0 \\
x+2y+2z&=1 \\
x-2y-2z&=0
\end{align}
Adding (1) and (2) will cancel out the $z$, giving us
\begin{align*}
10x+x+6y+2y&=0+1 \\
11x+8y&=1
\end{align*}
Using the value for $x$ we got above gives us
\begin{align*}
11\bigg(\displaystyle \frac{1}{2}\bigg)+8y&=1 \\
\displaystyle \frac{11}{2}+8y&=1 \\
8y&=1-\displaystyle \frac{11}{2} \\
8y&=\displaystyle \frac{2}{2}-\displaystyle \frac{11}{2} \\
8y&=-\displaystyle \frac{9}{2} \\
\displaystyle \frac{8y}{1}&=-\displaystyle \frac{9}{2} \\
8y\cdot2&=-9\cdot1 \\
16y&=-9 \\
y&=-\displaystyle \frac{9}{16}
\end{align*}
Using these values for $x$ and $y$ in the third equation (from the beginning) gives us
\begin{align*}
\displaystyle \frac{1}{2}-2\bigg(-\displaystyle \frac{9}{16}\bigg)-2z&=0 \\
\displaystyle \frac{1}{2}+\displaystyle \frac{18}{16}-2z&=0 \\
\displaystyle \frac{8}{16}+\displaystyle \frac{18}{16}-2z&=0 \\
\displaystyle \frac{26}{16}-2z&=0 \\
-2z&=-\displaystyle \frac{26}{16} \\
\displaystyle \frac{-2z}{1}&=-\displaystyle \frac{26}{16} \\
-2z\cdot16&=-26\cdot1 \\
-32z&=-26 \\
z&=\displaystyle \frac{-26}{-32} \\
z&=\displaystyle \frac{13}{16}
\end{align*}
\setcounter{equation}{0}
6. $\begin{aligned}[t]
2x+2y-3z&=0 \\
x-3y+z&=3 \\
2x+y-4z&=0
\end{aligned}$ \\
\\
We will reduce this to a system of two equations in two unknowns by eliminating the $x$. \\
We start by looking at the first and third equations. Multiplying the first equation by $-1$ gives us
\begin{align}
-2x-2y+3z&=0 \\
x-3y+z&=3 \\
2x+y-4z&=0
\end{align}
Adding (1) and (3) will cancel out the $x$, giving us
\begin{align}
-2y+y+3z-4z&=0+0 \\
-y-z&=0
\end{align}
Now we need another equation in two unknowns, so we go back to the beginning and eliminate $x$, this time looking at the second and third equations. Multiplying the second equation by $-2$ gives us
\begin{align}
2x+2y-3z&=0 \\
-2x+6y-2z&=-6 \\
2x+y-4z&=0
\end{align}
Adding (7) and (8) will cancel out the $x$, giving us
\begin{align}
6y+y-2z-4z&=-6+0 \\
7y-6z&=-6
\end{align}
Now we have a system of two equations (from (5) and (10))
\begin{align}
-y-z&=0 \\
7y-6z&=-6
\end{align}
Multiplying (11) by 7 gives us
\begin{align*}
-7y-7z&=0 \\
7y-6z&=-6
\end{align*}
Adding the two equations will cancel out the $y$, giving us
\begin{align*}
-7z-6z&=0-6 \\
-13z&=-6 \\
z&=\displaystyle \frac{-6}{-13} \\
z&=\displaystyle \frac{6}{13}
\end{align*}
Using this value for $z$ in (5) gives us
\begin{align*}
-y-\displaystyle \frac{6}{13}&=0 \\
-y&=\displaystyle \frac{6}{13} \\
y&=-\displaystyle \frac{6}{13}
\end{align*}
Using these values for $x$ and $y$ in the second equation (from the beginning) gives us
\begin{align*}
x-3\bigg(-\displaystyle \frac{6}{13}\bigg)+\displaystyle \frac{6}{13}&=3 \\
x+\displaystyle \frac{18}{13}+\displaystyle \frac{6}{13}&=3 \\
x+\displaystyle \frac{24}{13}&=3 \\
x&=3-\displaystyle \frac{24}{13} \\
x&=\displaystyle \frac{39}{13}-\displaystyle \frac{24}{13} \\
x&=\displaystyle \frac{15}{13}
\end{align*}
\setcounter{equation}{0}
7. $\begin{aligned}[t]
4x-2y+5z&=1 \\
x+y+z&=0 \\
-x+y-2z&=2
\end{aligned}$ \\
\\
We will reduce this to a system of two equations in two unknowns by eliminating the $x$. \\
We start by looking at the second and third equations. Adding the second and third equation will cancel out the $x$, giving us
\begin{align}
y+y+z-2z&=0+2 \\
2y-z&=2
\end{align}
Now we need another equation in two unknowns, so we go back to the beginning and eliminate the $x$, this time looking at the first and third equations. Multiplying the third equation by 4 gives us
\begin{align}
4x-2y+5z&=1 \\
x+y+z&=0 \\
-4x+4y-8z&=8
\end{align}
Adding (3) and (5) will cancel out the $x$, giving us
\begin{align}
-2y+4y+5z-8z&=1+8 \\
2y-3z&=9
\end{align}
Now we have a system of two equations (from (2) and (7))
\begin{align}
2y-z&=2 \\
2y-3z&=9
\end{align}
Multiplying (8) by $-1$ gives us
\begin{align*}
-2y+z&=-2 \\
2y-3z&=9
\end{align*}
Adding the two equations will cancel out the $y$, giving us
\begin{align*}
z-3z&=-2+9 \\
-2z&=7 \\
z&=-\displaystyle \frac{7}{2}
\end{align*}
Using this value for $z$ in (8) gives us
\begin{align*}
2y-\bigg(-\displaystyle \frac{7}{2}\bigg)&=2 \\
2y+\displaystyle \frac{7}{2}&=2 \\
2y&=2-\displaystyle \frac{7}{2} \\
2y&=\displaystyle \frac{4}{2}-\displaystyle \frac{7}{2} \\
2y&=\displaystyle \frac{-3}{2} \\
\displaystyle \frac{2y}{1}&=\displaystyle \frac{-3}{2} \\
2y\cdot2&=-3\cdot1 \\
4y&=-3 \\
y&=\displaystyle \frac{-3}{4}
\end{align*}
Using these values for $y$ and $z$ in the second equation (from the beginning) gives us
\begin{align*}
x-\displaystyle \frac{3}{4}-\displaystyle \frac{7}{2}&=0 \\
x-\displaystyle \frac{3}{4}-\displaystyle \frac{14}{4}&=0 \\
x-\displaystyle \frac{17}{4}&=0 \\
x&=\displaystyle \frac{17}{4}
\end{align*}
\setcounter{equation}{0}
8. $\begin{aligned}[t]
x+y+z&=0 \\
x-y-z&=1 \\
x+y-z&=1
\end{aligned}$ \\
\\
Normally, we would want to reduce this to a system of two equations in two unknowns. However, we can quickly find the value of $x$ by looking at the first and second equations. Adding the first and second equation will cancel out the $y$ and $z$, giving us
\begin{align*}
x+x+z-z&=0+1 \\
2x&=1 \\
x&=\displaystyle \frac{1}{2}
\end{align*}
Now we need an equation in two unknowns (with one of the unknowns being $x$), so we go back to the beginning, this time looking at the first and third equations. \\
Adding the first and third equation will cancel out the $z$, giving us
\begin{align*}
x+x+y+y&=0+1 \\
2x+2y&=1
\end{align*}
Using the value for $x$ we got above gives us
\begin{align*}
2\bigg(\displaystyle \frac{1}{2}\bigg)+2y&=1 \\
1+2y&=1 \\
2y&=0 \\
y&=0
\end{align*}
Using these values for $x$ and $y$ in the first equation (from the beginning) gives us
\begin{align*}
\displaystyle \frac{1}{2}+0+z&=0 \\
z&=-\displaystyle \frac{1}{2}
\end{align*}
\\
In the next exercises, you will find it easiest to clear denominators before solving. \\
\\
9. $\begin{aligned}[t]
\frac{1}{2}x+y-\frac{3}{4}z&=1 \\
\frac{2}{3}x-\frac{1}{3}y+z&=2 \\
x-\frac{1}{5}y+2z&=1
\end{aligned}$ \\
\\
To clear denominators, we multiply the first equation by 4, the second equation by 3, and the third equation by 5, which gives us
\begin{align}
2x+4y-3z&=4 \\
2x-y+3z&=6 \\
5x-y+10z&=5
\end{align}
We will reduce this to a system of two equations in two unknowns by eliminating the $y$. \\
We start by looking at (1) and (2). Multiplying (2) by 4 gives us
\begin{align}
2x+4y-3z&=4 \\
8x-4y+12z&=24 \\
5x-y+10z&=5
\end{align}
Adding (4) and (5) will cancel out the $y$, giving us
\begin{align}
2x+8x-3z+12z&=4+24 \\
10x+9z&=28
\end{align}
Now we need another equation in two unknowns, so we go back to the beginning and eliminate the $y$, this time looking at (2) and (3). Multiplying (2) by $-1$ gives us
\begin{align}
2x+4y-3z&=4 \\
-2x+y-3z&=-6 \\
5x-y+10z&=5
\end{align}
Adding (10) and (11) will cancel out the $y$, giving us
\begin{align}
-2x+5x-3z+10z&=-6+5 \\
3x+7z&=-1
\end{align}
Now we have a system of two equations (from (8) and (13))
\begin{align}
10x+9z&=28 \\
3x+7z&=-1
\end{align}
Multiplying (14) by 3 and (15) by $-10$ gives us
\begin{align*}
30x+27z&=84 \\
-30x-70z&=10
\end{align*}
Adding the two equations will cancel out the $z$, giving us
\begin{align*}
27z-70z&=84+10 \\
-43z&=94 \\
z&=-\displaystyle \frac{94}{43}
\end{align*}
Using this value for $z$ in (15) gives us
\begin{align*}
3x+7\bigg(-\displaystyle \frac{94}{43}\bigg)&=-1 \\
3x-\displaystyle \frac{658}{43}&=-1 \\
3x&=-1+\displaystyle \frac{658}{43} \\
3x&=\displaystyle \frac{43}{43}+\displaystyle \frac{658}{43} \\
3x&=\displaystyle \frac{615}{43} \\
\displaystyle \frac{3x}{1}&=\displaystyle \frac{615}{43} \\
3x\cdot43&=615\cdot1 \\
129x&=615 \\
x&=\displaystyle \frac{615}{129} \\
x&=\displaystyle \frac{205}{43}
\end{align*}
Using these values for $x$ and $z$ in (2) gives us
\begin{align*}
2\bigg(\displaystyle \frac{205}{43}\bigg)-y+3\bigg(-\displaystyle \frac{94}{43}\bigg)&=6 \\
\displaystyle \frac{410}{43}-y-\displaystyle \frac{282}{43}&=6 \\
-y+\displaystyle \frac{128}{43}&=6 \\
-y&=6-\displaystyle \frac{128}{43} \\
-y&=\displaystyle \frac{258}{43}-\displaystyle \frac{128}{43} \\
-y&=\displaystyle \frac{130}{43} \\
y&=-\displaystyle \frac{130}{43}
\end{align*}
\setcounter{equation}{0}
10. $\begin{aligned}[t]
\frac{1}{2}x-y+z&=1 \\
x+\frac{1}{3}y-\frac{2}{3}z&=2 \\
x+y-z&=3
\end{aligned}$ \\
\\
To clear denominators, we multiply the first equation by 2 and the second equation by 3, which gives us
\begin{align}
x-2y+2z&=2 \\
3x+y-2z&=6 \\
x+y-z&=3
\end{align}
Normally, we would want to reduce this to a system of two equations in two unknowns. However, we can quickly find the value of $x$ by looking at (1) and (3). Multiplying (3) by 2 gives us
\begin{align}
x-2y+2z&=2 \\
3x+y-2z&=6 \\
2x+2y-2z&=6
\end{align}
Adding (4) and (6) will cancel out the $y$ and $z$, giving us
\begin{align*}
x+2x&=2+6 \\
3x&=8 \\
x&=\displaystyle \frac{8}{3}
\end{align*}
Now we need an equation in two unknowns (with one of the unknowns being $x$), so we go back to the beginning, this time looking at (1) and (2). Adding (1) and (2) will cancel out the $z$, giving us
\begin{align}
x+3x-2y+y&=2+6 \\
4x-y&=8
\end{align}
Using the value we found for $x$ in (8) gives us
\begin{align*}
4\bigg(\displaystyle \frac{8}{3}\bigg)-y&=8 \\
\displaystyle \frac{32}{3}-y&=8 \\
-y&=8-\displaystyle \frac{32}{3} \\
-y&=\displaystyle \frac{24}{3}-\displaystyle \frac{32}{3} \\
-y&=-\displaystyle \frac{8}{3} \\
y&=\displaystyle \frac{8}{3}
\end{align*}
Using these values for $x$ and $y$ in (3) gives us
\begin{align*}
\displaystyle \frac{8}{3}+\displaystyle \frac{8}{3}-z&=3 \\
\displaystyle \frac{16}{3}-z&=3 \\
-z&=3-\displaystyle \frac{16}{3} \\
-z&=\displaystyle \frac{9}{3}-\displaystyle \frac{16}{3} \\
z&=\displaystyle \frac{7}{3}
\end{align*}
\setcounter{equation}{0}
11. $\begin{aligned}[t]
\frac{3}{4}x-y+z&=1 \\
x-\frac{1}{2}y+z&=0 \\
x+y-\frac{1}{3}z&=1
\end{aligned}$ \\
\\
To clear denominators, we multiply the first equation by 4, the second equation by 2, and the third equation by 3, which gives us
\begin{align}
3x-4y+4z&=4 \\
2x-y+2z&=0 \\
3x+3y-z&=3
\end{align}
We will reduce this to a system of two equations in two unknowns by eliminating the $z$. \\
We start by looking at (2) and (3). Multiplying (3) by 2 gives us
\begin{align}
3x-4y+4z&=4 \\
2x-y+2z&=0 \\
6x+6y-2z&=6
\end{align}
Adding (5) and (6) will cancel out the $z$, giving us
\begin{align}
2x+6x-y+6y&=0+6 \\
8x+5y&=6
\end{align}
Now we need another equation in two unknowns, so we go back to the beginning and eliminate the $z$, this time looking at (1) and (3). Multiplying (3) by 4 gives us
\begin{align}
3x-4y+4z&=4 \\
2x-y+2z&=0 \\
12x+12y-4z&=12
\end{align}
Adding (9) and (11) will cancel out the $z$, giving us
\begin{align}
3x+12x-4y+12y&=4+12 \\
15x+8y&=16
\end{align}
Now we have a system of two equations (from (8) and (13))
\begin{align}
8x+5y&=6 \\
15x+8y&=16
\end{align}
Multiplying (14) by $-8$ and (15) by 5 gives us
\begin{align*}
-64x-40y&=-48 \\
75x+40y&=80
\end{align*}
Adding the two equations will cancel out the $y$, giving us
\begin{align*}
-64x+75x&=-48+80 \\
11x&=32 \\
x&=\displaystyle \frac{32}{11}
\end{align*}
Using this value for $x$ in (14) gives us
\begin{align*}
8\bigg(\displaystyle \frac{32}{11}\bigg)+5y&=6 \\
\displaystyle \frac{256}{11}+5y&=6 \\
5y&=6-\displaystyle \frac{256}{11} \\
5y&=\displaystyle \frac{66}{11}-\displaystyle \frac{256}{11} \\
5y&=-\displaystyle \frac{190}{11} \\
\displaystyle \frac{5y}{1}&=-\displaystyle \frac{190}{11} \\
5y\cdot11&=-190\cdot1 \\
55y&=-190 \\
y&=\displaystyle \frac{-190}{55} \\
y&=-\displaystyle \frac{38}{11}
\end{align*}
Using these values for $x$ and $y$ in (3) gives us
\begin{align*}
3\bigg(\displaystyle \frac{32}{11}\bigg)+3\bigg(-\displaystyle \frac{38}{11}\bigg)-z&=3 \\
\displaystyle \frac{96}{11}-\displaystyle \frac{114}{11}-z&=3 \\
-\displaystyle \frac{18}{11}-z&=3 \\
-z&=3+\displaystyle \frac{18}{11} \\
-z&=\displaystyle \frac{33}{11}+\displaystyle \frac{18}{11} \\
-z&=\displaystyle \frac{51}{11} \\
z&=-\displaystyle \frac{51}{11}
\end{align*}
\setcounter{equation}{0}
12. $\begin{aligned}[t]
\frac{1}{2}x-\frac{2}{3}y+z&=1 \\
x-\frac{1}{5}y+z&=0 \\
2x-\frac{1}{3}y+\frac{2}{5}z&=1
\end{aligned}$ \\
\\
To clear denominators, we multiply the first equation by 6, the second equation by 5, and the third equation by 15, which gives us
\begin{align}
3x-4y+6z&=6 \\
5x-y+5z&=0 \\
30x-5y+6z&=15
\end{align}
We will reduce this to a system of two equations in two unknowns by eliminating the $y$. \\
We start by looking at (1) and (2). Multiplying (2) by $-4$ gives us
\begin{align}
3x-4y+6z&=6 \\
-20x+4y-20z&=0 \\
30x-5y+6z&=15
\end{align}
Adding (4) and (5) will cancel out the $y$, giving us
\begin{align}
3x-20x+6z-20z&=6+0 \\
-17x-14z&=6
\end{align}
Now we need another equation in two unknowns, so we go back to the beginning and eliminate the $y$, this time looking at (2) and (3). Multiplying (2) by $-5$ gives us
\begin{align}
3x-4y+6z&=6 \\
-25x+5y-25z&=0 \\
30x-5y+6z&=15
\end{align}
Adding (10) and (11) will cancel out the $y$, giving us
\begin{align}
-25x+30x-25z+6z&=0+15 \\
5x-19z&=15
\end{align}
Now we have a system of two equations (from (8) and (13))
\begin{align}
-17x-14z&=6 \\
5x-19z&=15
\end{align}
Multiplying (14) by 5 and (15) by 17 gives us
\begin{align*}
-85x-70z&=30 \\
85x-323z&=255
\end{align*}
Adding the two equations will cancel out the $x$, giving us
\begin{align*}
-70z-323z&=30+255 \\
-393z&=285 \\
z&=\displaystyle \frac{285}{-393} \\
z&=\displaystyle \frac{95}{-131}
\end{align*}
Using this value for $z$ in (14) gives us
\begin{align*}
-17x-14\bigg(-\displaystyle \frac{95}{131}\bigg)&=6 \\
-17x+\displaystyle \frac{1330}{131}&=6 \\
-17x&=6-\displaystyle \frac{1330}{131} \\
-17x&=\displaystyle \frac{786}{131}-\displaystyle \frac{1330}{131} \\
-17x&=-\displaystyle \frac{544}{131} \\
-\displaystyle \frac{17x}{1}&=-\displaystyle \frac{544}{131} \\
-17x\cdot131&=-544\cdot1 \\
-2227x&=-544 \\
x&=\displaystyle \frac{-544}{-2227} \\
x&=\displaystyle \frac{32}{131}
\end{align*}
Using these values for $x$ and $z$ in (2) gives us
\begin{align*}
5\bigg(\displaystyle \frac{32}{131}\bigg)-y+5\bigg(-\displaystyle \frac{95}{131}\bigg)&=0 \\
\displaystyle \frac{160}{131}-y-\displaystyle \frac{475}{131}&=0 \\
-y-\displaystyle \frac{315}{131}&=0 \\
-y&=\displaystyle \frac{315}{131} \\
y&=-\displaystyle \frac{315}{131}
\end{align*}
\end{document}
